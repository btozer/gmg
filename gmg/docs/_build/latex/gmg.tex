%% Generated by Sphinx.
\def\sphinxdocclass{report}
\documentclass[a4paper,10pt,english]{sphinxmanual}
\ifdefined\pdfpxdimen
   \let\sphinxpxdimen\pdfpxdimen\else\newdimen\sphinxpxdimen
\fi \sphinxpxdimen=.75bp\relax

\usepackage[utf8]{inputenc}
\ifdefined\DeclareUnicodeCharacter
 \ifdefined\DeclareUnicodeCharacterAsOptional
  \DeclareUnicodeCharacter{"00A0}{\nobreakspace}
  \DeclareUnicodeCharacter{"2500}{\sphinxunichar{2500}}
  \DeclareUnicodeCharacter{"2502}{\sphinxunichar{2502}}
  \DeclareUnicodeCharacter{"2514}{\sphinxunichar{2514}}
  \DeclareUnicodeCharacter{"251C}{\sphinxunichar{251C}}
  \DeclareUnicodeCharacter{"2572}{\textbackslash}
 \else
  \DeclareUnicodeCharacter{00A0}{\nobreakspace}
  \DeclareUnicodeCharacter{2500}{\sphinxunichar{2500}}
  \DeclareUnicodeCharacter{2502}{\sphinxunichar{2502}}
  \DeclareUnicodeCharacter{2514}{\sphinxunichar{2514}}
  \DeclareUnicodeCharacter{251C}{\sphinxunichar{251C}}
  \DeclareUnicodeCharacter{2572}{\textbackslash}
 \fi
\fi
\usepackage{cmap}
\usepackage[T1]{fontenc}
\usepackage{amsmath,amssymb,amstext}
\usepackage{babel}
\usepackage{times}
\usepackage[Bjarne]{fncychap}
\usepackage[dontkeepoldnames]{sphinx}

\usepackage{geometry}

% Include hyperref last.
\usepackage{hyperref}
% Fix anchor placement for figures with captions.
\usepackage{hypcap}% it must be loaded after hyperref.
% Set up styles of URL: it should be placed after hyperref.
\urlstyle{same}

\addto\captionsenglish{\renewcommand{\figurename}{Fig.}}
\addto\captionsenglish{\renewcommand{\tablename}{Table}}
\addto\captionsenglish{\renewcommand{\literalblockname}{Listing}}

\addto\captionsenglish{\renewcommand{\literalblockcontinuedname}{continued from previous page}}
\addto\captionsenglish{\renewcommand{\literalblockcontinuesname}{continues on next page}}

\addto\extrasenglish{\def\pageautorefname{page}}

\setcounter{tocdepth}{0}



\title{gmg Documentation}
\date{Aug 22, 2018}
\release{1}
\author{B. Tozer}
\newcommand{\sphinxlogo}{\sphinxincludegraphics{gmg_icon.png}\par}
\renewcommand{\releasename}{Release}
\makeindex

\begin{document}

\maketitle
\sphinxtableofcontents
\phantomsection\label{\detokenize{gmg_documentation::doc}}



\chapter{Overview}
\label{\detokenize{gmg_documentation:gmg-an-open-source-geophysical-modelling-gui}}\label{\detokenize{gmg_documentation:overview}}
GMG is designed primarily as a Graphical User Interface (GUI) for 2D forward
modelling of gravity and magnetic data. GMG can also load SEGY, well horizons
and XY point data, providing a full 2D geologic/geophysical profile
interpretation package.

GMG is written in python 2.7 using wx.python for GUI implementation. GMG makes
use of several other open source python packages to preform various tasks
(see \sphinxhref{references.html}{References} for a detailed listing).


\chapter{Index}
\label{\detokenize{gmg_documentation:index}}

\section{\sphinxstylestrong{Installation}}
\label{\detokenize{installation:id1}}\label{\detokenize{installation::doc}}\label{\detokenize{installation:installation}}
\begin{sphinxadmonition}{tip}{Tip:}
Following the steps listed below should ensure everything that is required to run gmg is installed on your system correctly.
\end{sphinxadmonition}


\subsection{Operating System}
\label{\detokenize{installation:operating-system}}
\begin{sphinxadmonition}{note}{Note:}
gmg is (for the time being) is only tested and supported on the Ubuntu 14.04 LINUX operating system.
Issues may arise when trying to install on other distributions.
\end{sphinxadmonition}


\subsection{Dependencies}
\label{\detokenize{installation:dependencies}}
GMG depends on several other packages to run. These are:
\begin{itemize}
\item {} 
\sphinxhref{http://numpy.scipy.org/}{numpy}

\item {} 
\sphinxhref{http://scipy.org/}{scipy}

\item {} 
\sphinxhref{http://matplotlib.sourceforge.net/}{matplotlib}

\item {} 
\sphinxhref{http://www.fatiando.org/}{fatiando a terra}

\item {} 
\sphinxhref{http://wiki.wxpython.org/}{wxpython}

\item {} 
\sphinxhref{http://docs.obspy.org/}{ObsPy}

\end{itemize}

Each dependency needs to be installed in the correct directory in order for GMG to run. Following the steps below is
the simplest way to compete the installation.


\subsection{Step 1:}
\label{\detokenize{installation:step-1}}
The simplest way to get started is to follow the instructions for installing fatiando via the Anaconda
python package manager as described here:

\sphinxhref{http://www.fatiando.org/install.html/}{Install fatiando}

As explained in the link above, Anaconda is a compilation of python packages/package
managing tool that can be used to install additional packages including fatiando, numpy, scipy and matplotlib.


\subsection{Step 2:}
\label{\detokenize{installation:step-2}}
Install wxPython. On Ubuntu linux the simplest way to ensure you have everything that is required
is to copy the command below and paste in the command line of a terminal:

\fvset{hllines={, ,}}%
\begin{sphinxVerbatim}[commandchars=\\\{\}]
sudo apt\PYGZhy{}get install python\PYGZhy{}wxgtk2.8 python\PYGZhy{}wxtools wx2.8\PYGZhy{}i18n libwxgtk2.8\PYGZhy{}dev libgtk2.0\PYGZhy{}dev
\end{sphinxVerbatim}


\subsection{Step 3:}
\label{\detokenize{installation:step-3}}
Unzip the gmg package into any directory e.g. \sphinxstyleemphasis{\textasciitilde{}/Downloads}

Open a terminal and navigate to the directory that contains your fatiando installation.
This is likely to be:

\fvset{hllines={, ,}}%
\begin{sphinxVerbatim}[commandchars=\\\{\}]
cd \PYGZti{}/anaconda/lib/python2.7/site\PYGZhy{}packages/fatiando
\end{sphinxVerbatim}

Move the gui directory from gmg into the fatiando directory e.g.:

\fvset{hllines={, ,}}%
\begin{sphinxVerbatim}[commandchars=\\\{\}]
mv \PYGZti{}/Downloads/gmg/gui .
\end{sphinxVerbatim}

This directory should contain:
\begin{itemize}
\item {} \begin{enumerate}
\item {} 
The script: forward.py (which contains the gmg package)

\end{enumerate}

\item {} \begin{enumerate}
\setcounter{enumi}{1}
\item {} 
The script: plot\_model.py

\end{enumerate}

\item {} \begin{enumerate}
\setcounter{enumi}{2}
\item {} 
The script: model\_stats.py

\end{enumerate}

\item {} \begin{enumerate}
\setcounter{enumi}{2}
\item {} 
A directory called icons

\end{enumerate}

\end{itemize}

Navigate to the fatiando gravmag directory using the command:

\fvset{hllines={, ,}}%
\begin{sphinxVerbatim}[commandchars=\\\{\}]
cd gravmag
\end{sphinxVerbatim}

From the gmg directory \sphinxstyleemphasis{gmg/gravmag} copy the following into the \sphinxstyleemphasis{fatiando/gravmag} directory e.g.:

\fvset{hllines={, ,}}%
\begin{sphinxVerbatim}[commandchars=\\\{\}]
cp \PYGZti{}/Downloads/gmg/gravmag/*.py .
\end{sphinxVerbatim}

This will copy:
\begin{itemize}
\item {} \begin{enumerate}
\item {} 
The script: talwani.py (NOTE: This will overwrite the default fatiando talwani.py script with some edits)

\end{enumerate}

\item {} \begin{enumerate}
\setcounter{enumi}{1}
\item {} 
The script: won\_and\_bevis\_mag2d.py

\end{enumerate}

\end{itemize}


\subsection{Step 4:}
\label{\detokenize{installation:step-4}}
In the gmg directory make the \sphinxstyleemphasis{GMG.py} script executable using the command:

\fvset{hllines={, ,}}%
\begin{sphinxVerbatim}[commandchars=\\\{\}]
chmod 777 gmg.py
\end{sphinxVerbatim}

Now move gmg.py into you \sphinxstyleemphasis{/bin} directory or any other directory that is within your \sphinxstyleemphasis{\$PATH} variable e.g.:

\fvset{hllines={, ,}}%
\begin{sphinxVerbatim}[commandchars=\\\{\}]
sudo mv GMG.py /bin/GMG.py
\end{sphinxVerbatim}


\subsection{DONE!}
\label{\detokenize{installation:done}}
You should now be able to launch gmg from a terminal by typing \sphinxstyleemphasis{GMG}

\begin{sphinxadmonition}{tip}{Tip:}
You can also edit your .bashrc file and create a custom launch command for GMG by adding an alias such as:

\fvset{hllines={, ,}}%
\begin{sphinxVerbatim}[commandchars=\\\{\}]
alias GMG=\PYGZsq{}python /bin/GMG.py\PYGZsq{}
\end{sphinxVerbatim}
\end{sphinxadmonition}


\section{\sphinxstylestrong{Getting Started}}
\label{\detokenize{getting_started:id1}}\label{\detokenize{getting_started:getting-started}}\label{\detokenize{getting_started::doc}}

\subsection{1. Launching the software:}
\label{\detokenize{getting_started:launching-the-software}}
Open a new terminal and launch gmg using the command gmg.py or the alias you may have setup.


\subsection{2. Creating a new model:}
\label{\detokenize{getting_started:creating-a-new-model}}
Navigate to the \sphinxstyleemphasis{file} menu and select \sphinxstyleemphasis{new model…}

Here you input:
* 1. The new model dimensions in km
* 2. The spacing at which potential fields are to be calculated.

\begin{sphinxadmonition}{note}{Note:}
Setting this spacing as a relatively coarse value to begin with will ensure the program runs smoothly,
this can later be refined for detailed analysis once a first model is complete.
\end{sphinxadmonition}


\subsection{3. Get Modelling!}
\label{\detokenize{getting_started:get-modelling}}
You can now add layers and edit the model.

See the Tools instructions for details on how to do this.


\section{\sphinxstylestrong{Manual}}
\label{\detokenize{manual:gui-manual}}\label{\detokenize{manual:manual}}\label{\detokenize{manual::doc}}

\subsection{\sphinxstylestrong{Icon Shortcuts}}
\label{\detokenize{manual_icons:icons}}\label{\detokenize{manual_icons:icon-shortcuts}}\label{\detokenize{manual_icons::doc}}

\begin{savenotes}\sphinxattablestart
\centering
\begin{tabulary}{\linewidth}[t]{|T|T|}
\hline

\sphinxincludegraphics{{1_save}.png} Save model
&
\sphinxincludegraphics{{1_load}.png} Load model
\\
\hline&\\
\hline
\sphinxincludegraphics{{1_g}.png} Compute gravity anomaly
&
\sphinxincludegraphics{{1_m}.png} Compute magnetic anomaly
\\
\hline&\\
\hline
\sphinxincludegraphics{{1_c}.png} Capture Coordinates
&
\sphinxincludegraphics{{1_fault}.png} Start fault picking
\\
\hline&\\
\hline
\sphinxincludegraphics{{1_up}.png} Increase aspect
&
\sphinxincludegraphics{{1_down}.png} Decrease aspect
\\
\hline&\\
\hline
\sphinxincludegraphics{{1_up2}.png} Increase aspect x2
&
\sphinxincludegraphics{{1_down2}.png} Decrease aspect x2
\\
\hline&\\
\hline
\sphinxincludegraphics{{1_zoom_in}.png} Zoom in
&
\sphinxincludegraphics{{1_zoom_out}.png} Zoom out
\\
\hline&\\
\hline
\sphinxincludegraphics{{1_extent}.png} Full extent
&
\sphinxincludegraphics{{1_pan}.png} Pan
\\
\hline&\\
\hline
\sphinxincludegraphics{{1_down_h}.png} Segy gain up
&
\sphinxincludegraphics{{1_up_h}.png} Segy gain down
\\
\hline&\\
\hline
\sphinxincludegraphics{{1_down2_h}.png} Transparency increase
&
\sphinxincludegraphics{{1_up2_h}.png} Transparency decrease
\\
\hline&\\
\hline
\sphinxincludegraphics{{1_well}.png} Load well data
&\\
\hline
\end{tabulary}
\par
\sphinxattableend\end{savenotes}


\subsection{\sphinxstylestrong{Keyboard Shortcuts}}
\label{\detokenize{manual_keyboard_shortcuts:keyboard-shortcuts}}\label{\detokenize{manual_keyboard_shortcuts:keys}}\label{\detokenize{manual_keyboard_shortcuts::doc}}

\begin{savenotes}\sphinxattablestart
\centering
\begin{tabulary}{\linewidth}[t]{|T|T|}
\hline
\sphinxstylethead{\sphinxstyletheadfamily 
Key
\unskip}\relax &\sphinxstylethead{\sphinxstyletheadfamily 
Function
\unskip}\relax \\
\hline
\sphinxstylestrong{i}
&
Insert Node
\\
\hline
\sphinxstylestrong{d}
&
Delete Node
\\
\hline
\sphinxstylestrong{p}
&
Pinch Node\textasciicircum{}
\\
\hline
\sphinxstylestrong{\textgreater{}}
&
Next Layer
\\
\hline
\sphinxstylestrong{\textless{}}
&
Previous Layer
\\
\hline
\sphinxstylestrong{z}
&
Zoom
\\
\hline
\sphinxstylestrong{a}
&
Show All (1:1 aspect)
\\
\hline
\sphinxstylestrong{shift+p}
&
Pan
\\
\hline
\end{tabulary}
\par
\sphinxattableend\end{savenotes}

\textasciicircum{}Left click selects node to pinch, now change layer and select the node to pinch onto.


\subsection{\sphinxstylestrong{1.0 Models}}
\label{\detokenize{manual_models:models}}\label{\detokenize{manual_models::doc}}
\begin{DUlineblock}{0em}
\item[] 
\end{DUlineblock}


\subsubsection{1.1 Creating a New Model}
\label{\detokenize{manual_models:creating-a-new-model}}
Before creating a new model, it is recommended that you create a directory structure in which you will store and save
all of the data related to the model. This can be structured in any way the user wishes but the directory structure
as outlined below is useful:

\fvset{hllines={, ,}}%
\begin{sphinxVerbatim}[commandchars=\\\{\}]
mkdir *model\PYGZhy{}name*\PYGZus{}MODEL
\end{sphinxVerbatim}

Then:

\fvset{hllines={, ,}}%
\begin{sphinxVerbatim}[commandchars=\\\{\}]
cd model\PYGZhy{}name\PYGZus{}MODEL
\end{sphinxVerbatim}

and create the following directory structure:

\fvset{hllines={, ,}}%
\begin{sphinxVerbatim}[commandchars=\\\{\}]
mkdir MODELS GRAVITY\PYGZus{}DATA MAGNETIC\PYGZus{}DATA SEGY\PYGZus{}DATA WELL\PYGZus{}DATA GEOLOGICAL\PYGZus{}DATA POINT\PYGZus{}DATA EXPORTS FIGURES
\end{sphinxVerbatim}

Now you are ready to create your model. Launch gmg and navigate to:

File -\textgreater{} New Model…

You will be prompted by a window in which to enter the new model’s dimensions and the spacing
increment at which potential field data is to be calculated (all units are km).

\begin{sphinxadmonition}{note}{Note:}
The model dimensions and spacing at which potential field data is calculated can be modified later by using the
Model View -\textgreater{} Modify current Model Dimensions… option.
\end{sphinxadmonition}

\begin{sphinxadmonition}{tip}{Tip:}
If a very fine spatial resolution relative to the length of the model is required for predicted anomalies, it may
be computationally beneficial to set the  spacing at which potential field data is to be calculated as a relatively
coarse value for initial modelling and later reduce this to the fine spacing required for the final model. This will
ensure GMG runs smoothly during the initial forward modelling.
\end{sphinxadmonition}

\begin{DUlineblock}{0em}
\item[] 
\end{DUlineblock}


\subsubsection{1.2 Saving a Model}
\label{\detokenize{manual_models:saving-a-model}}
To save a model navigate to Files -\textgreater{} Save Model…
This is prompt a save file menu from which you can navigate to the directory where you would like to save your model,
for example gmg\_MODEL/MODELS/

gmg models are saved as python Pickle files and must have a “.model” suffix, for example,
gmg\_MODEL/MODELS/gmg\_model\_01.model

\begin{sphinxadmonition}{tip}{Tip:}
It is recommended that the user incrementally saves updated models with a new suffix such as model\_1.model,
model\_02.model…etc
\end{sphinxadmonition}

This will ensure you can revert back to previous models if your current model becomes a “dead end”.

\begin{DUlineblock}{0em}
\item[] 
\end{DUlineblock}


\subsubsection{1.2 Loading a Model}
\label{\detokenize{manual_models:loading-a-model}}
To load a model navigate to Files -\textgreater{} Load Model…
Navigate to your required .model and select Open.

Your model will be loaded into the current gmg window.


\subsection{\sphinxstylestrong{2.0 Model Layers}}
\label{\detokenize{manual_layer_nodes:model-layers}}\label{\detokenize{manual_layer_nodes::doc}}
\begin{DUlineblock}{0em}
\item[] 
\end{DUlineblock}


\subsubsection{2.1 Adding a layer}
\label{\detokenize{manual_layer_nodes:adding-a-layer}}
To add a new layer to the model navigate to the \sphinxstyleemphasis{‘Layers’} menu and select \sphinxstyleemphasis{‘New Layer’}.

\begin{sphinxadmonition}{tip}{Tip:}
the \sphinxstyleemphasis{n} key is the keyboard shortcut for adding a new layer.
\end{sphinxadmonition}

You will be prompted with the option of adding a \sphinxstyleemphasis{‘New fixed layer’} or a \sphinxstyleemphasis{‘New floating layer’}.

\sphinxstylestrong{Floating layers:}

Floating layers are polygons that are fully contained within the model space.

\sphinxstylestrong{Fixed layers:}

Fixed layers span the entire width of the model and are useful for modelling features such as sedimentary basins.
Fixed layers have fixed boundary nodes that can only be moved vertically along the model edges.
These layers are padded horizontally from the boundary nodes an additional 400 km in order to avoid edge effects.

\begin{DUlineblock}{0em}
\item[] 
\end{DUlineblock}


\subsubsection{2.2 Adding a New Node to a Layer}
\label{\detokenize{manual_layer_nodes:adding-a-new-node-to-a-layer}}
To add a node to a layer place the mouse cursor at the position where you would like to add the new node and
press the \sphinxstyleemphasis{‘i’} key to insert the new node.

\begin{sphinxadmonition}{note}{Note:}
Nodes are inserted to the right of the currently selected node.
\end{sphinxadmonition}

\begin{DUlineblock}{0em}
\item[] 
\end{DUlineblock}


\subsubsection{2.3 Deleting a Node From a Layer}
\label{\detokenize{manual_layer_nodes:deleting-a-node-from-a-layer}}
To delete a node from a layer place the mouse cursor over the node you would like to delete and
press the \sphinxstyleemphasis{‘d’} key to delete the new node.

\begin{DUlineblock}{0em}
\item[] 
\end{DUlineblock}


\subsubsection{2.4 Pinching nodes}
\label{\detokenize{manual_layer_nodes:pinching-nodes}}
To “pinch” a node onto a node of another layer: Press the \sphinxstyleemphasis{p} key to activate pinch mode.
Now select the node you wish to pinch, then switch to the layer which you would like to pinch
to, finally click on the node to pinch to.

Alternatively, multiple nodes can be pinched or depinched simultaneously to the layer above or below
(as defined in the layer order list) by selecting \sphinxstyleemphasis{‘pinch layer’} in the Menubar.
A distance range over which to pinch/depinch nodes is then required.


\subsection{\sphinxstylestrong{3.0 Layer attributes}}
\label{\detokenize{manual_layer_attributes:layer-attributes}}\label{\detokenize{manual_layer_attributes::doc}}
\begin{DUlineblock}{0em}
\item[] 
\end{DUlineblock}


\subsubsection{3.1 Density}
\label{\detokenize{manual_layer_attributes:density}}
Gravity modelling is achieved using bulk \sphinxstyleemphasis{density contrasts} relative to a \sphinxstyleemphasis{reference density}.

Each layer requires:
\begin{itemize}
\item {} 
An absolute bulk density

\item {} 
A reference density

\end{itemize}

All densities must be input using units \(kg/m^3\).

These are input using the attribute side bar or within the \sphinxstyleemphasis{Attribute table}.

This enables gravity anomalies relative to say an upper crust of 2670 \(kg/m^3\), lower crust 2900 \(kg/m^3\)
and upper mantle of 3330 \(kg/m^3\).

For each layer both the bulk absolute density and the reference density must be set. For example:

To model a sedimentary unit it may have a density set as 2300 \(kg/m^3\) and reference crustal
density of 2670 \(kg/m^3\).

To model crustal thickened from a reference crustal thickness of 32 km to 36 km, a lower crustal density of
2800 \(kg/m^3\) maybe be modelled against a upper most mantle reference density of 3330 \(kg/m^3\).

\begin{sphinxadmonition}{tip}{Tip:}
To set all layer reference densities as a single value (e.g. when only modelling upper crustal structure)
\end{sphinxadmonition}

use the \sphinxstyleemphasis{gravity Field} -\textgreater{} \sphinxstyleemphasis{Set background density} tool.

\begin{DUlineblock}{0em}
\item[] 
\end{DUlineblock}


\subsubsection{3.2 Magnetic Susceptibility}
\label{\detokenize{manual_layer_attributes:magnetic-susceptibility}}
Susceptibility must be input using SI units.

A ‘\sphinxstyleemphasis{strike}’ must also be assigned for each unit.

\begin{sphinxadmonition}{important}{Important:}
\sphinxstyleemphasis{Strike} is the angle (in degrees) that the strike of the unit in map view makes with respect to magnetic north
(the angle is positive when measured counterclockwise) See Won and Bevis (1987) for a diagram and further details.
The strike of the unit is assumed to be orthogonal to the model transect (striking into and out of the screen).

For example:

If the model is orientated West-East and magnetic north is -22 deg the strike of the unit is -22 deg.

If the model was orientated North-South the angle would be 68 deg.
\end{sphinxadmonition}

\begin{DUlineblock}{0em}
\item[] 
\end{DUlineblock}


\subsubsection{3.3 Remenant Magnetism}
\label{\detokenize{manual_layer_attributes:remenant-magnetism}}

\subsection{\sphinxstylestrong{4.0 Potential field data}}
\label{\detokenize{manual_PF:potential-field-data}}\label{\detokenize{manual_PF::doc}}
\begin{DUlineblock}{0em}
\item[] 
\end{DUlineblock}


\subsubsection{\sphinxstylestrong{4.1 Potential field calculations}}
\label{\detokenize{manual_PF:potential-field-calculations}}
The Menubar contains buttons labelled \sphinxstyleemphasis{G} and \sphinxstyleemphasis{M}.
These are used for switching the calculations of the potential fields on and off. Turing off the predicted anomalies
can help speed up the GUI response time if the model becomes complex (many layers and nodes).

\begin{sphinxadmonition}{note}{Note:}
You must set the magnetic inclination and Earth Field value under the \sphinxstyleemphasis{Magnetic Field} menu before
\end{sphinxadmonition}

the predicted anomaly can be calculated.

\begin{DUlineblock}{0em}
\item[] 
\end{DUlineblock}


\subsubsection{\sphinxstylestrong{4.2 Loading observed potential field data}}
\label{\detokenize{manual_PF:loading-observed-potential-field-data}}
To load observed potential field data click on the \sphinxstyleemphasis{Observed} menu in the Menubar and select either
\sphinxstyleemphasis{Load observed Gravity anomaly} or \sphinxstyleemphasis{Load observed Magnetic anomaly}.
Next Navigate to the file you want to load and select it.
The file will automatically be loaded into the canvas.

\begin{sphinxadmonition}{note}{Note:}
These files should be ASCII text files with X values in the first column and the anomaly value
\end{sphinxadmonition}

in the second column. These files may have any suffix.


\subsection{\sphinxstylestrong{5.0 SEGY data}}
\label{\detokenize{manual_segy::doc}}\label{\detokenize{manual_segy:segy-data}}

\subsubsection{\sphinxstylestrong{5.1 Loading SEGY data}}
\label{\detokenize{manual_segy:loading-segy-data}}
To load observed SEGY data select the \sphinxstyleemphasis{‘SEGY’} menu in the Menubar.
Next Navigate to the file you want to load and select it.
The file will automatically be loaded into the canvas.

\begin{sphinxadmonition}{note}{Note:}
SEGY data is loaded using the obspy.SEGY.core routine. This requires that all time/date SEGY headers have values,
so these may need to be set using seismic processing software or obspy if they are not present in your data.

The SEGY file will be plotted as a pyplot image.
\end{sphinxadmonition}


\subsection{\sphinxstylestrong{6.0 Well data}}
\label{\detokenize{manual_wells:well-data}}\label{\detokenize{manual_wells::doc}}

\subsubsection{\sphinxstylestrong{6.1 Loading well data}}
\label{\detokenize{manual_wells:loading-well-data}}
To load observed Well horizons select the \sphinxstyleemphasis{Well} menu in the Menubar and select \sphinxstyleemphasis{Load..*}. Or select the load well icon.
Navigate to the file you wish to load.

\begin{sphinxadmonition}{note}{Note:}
These files should be ASCII text files formatted as shown in the \sphinxstyleemphasis{example1.well} well file.
\end{sphinxadmonition}

Once well horizons are loaded into your model, you can increase and decrease the label text size using the slider in the
left hand menu. Each well can be hidden/shown by selecting the well name under \sphinxstyleemphasis{Well…} in the \sphinxstyleemphasis{Wells} menu.


\subsection{\sphinxstylestrong{9.0 XY Data}}
\label{\detokenize{manual_XY_data::doc}}\label{\detokenize{manual_XY_data:xy-data}}

\subsection{\sphinxstylestrong{7.0 Geologic data}}
\label{\detokenize{manual_geology:geologic-data}}\label{\detokenize{manual_geology::doc}}

\subsubsection{\sphinxstylestrong{7.1 Loading geologic data}}
\label{\detokenize{manual_geology:loading-geologic-data}}

\subsection{\sphinxstylestrong{8.0 Export Figure}}
\label{\detokenize{manual_export_figure:export-figure}}\label{\detokenize{manual_export_figure::doc}}

\subsection{** Python console**}
\label{\detokenize{manual_python_console:python-console}}\label{\detokenize{manual_python_console::doc}}

\section{\sphinxstylestrong{Tutorial}}
\label{\detokenize{tutorial:tutorial}}\label{\detokenize{tutorial::doc}}

\section{\sphinxstylestrong{Contribute}}
\label{\detokenize{contribute:contibute}}\label{\detokenize{contribute:contribute}}\label{\detokenize{contribute::doc}}

\section{\sphinxstylestrong{References}}
\label{\detokenize{references:references}}\label{\detokenize{references:id1}}\label{\detokenize{references::doc}}
\sphinxstylestrong{Fatiando a Terra}

Uieda, L, Oliveira Jr, V C, Ferreira, A, Santos, H B; Caparica Jr, J F (2014), Fatiando a Terra: a Python package for
modeling and inversion in geophysics. figshare. doi:10.6084/m9.figshare.1115194

\sphinxstylestrong{Matplotlib}

Hunter, J. D. (2007). Matplotlib: A 2D graphics environment. Computing In Science \& Engineering, 9(3),
90\textendash{}95. doi:10.5281/zenodo.15423

\sphinxstylestrong{SciPy}

Oliphant, T. E. (2007). Python for Scientific Computing Python Overview. Computing In Science \& Engineering, 10-20.

\sphinxstylestrong{NumPy}

van der Walt, S., Colbert, S. C., and Varoquaux, G. (2011). The NumPy Array: A Structure for Efficient
Numerical Computation. Computing in Science \& Engineering, 13(2):22-30.

\sphinxstylestrong{wxPython}

Rappin, N. and Dunn, R. (2006). wxPython in Action. Manning Publications.

\sphinxstylestrong{ObsPy}

SEGY viewer from obspy package. obspy.org

Beyreuther, M. R. Barsch, L. Krischer, T. Megies, Y. Behr and J. Wassermann (2010) ObsPy: a python toolbox for
seismology Seismological Research Letters, 81(3), 530-533

\sphinxstylestrong{Gravity Algorithm}

Bott, M. H. P. (1969). GRAVN. Durham geophysical computer specification No. 1.

\sphinxstylestrong{Magnetic Algorithm}

Talwani, M., \& Heirtzler, J. R. (1964). Computation of magnetic anomalies caused by two dimensional structures of
arbitrary shape, in Parks, G. A., Ed., Computers in the mineral industries, Part 1: Stanford Univ. Publ., Geological
Sciences, 9, 464-480.

\sphinxstylestrong{Icons}

Icons designed using: \sphinxurl{https://freeiconmaker.com/}


\section{\sphinxstylestrong{Licencing}}
\label{\detokenize{licence:licencing}}\label{\detokenize{licence:licence}}\label{\detokenize{licence::doc}}

\subsection{gmg Licence}
\label{\detokenize{licence:gmg-licence}}
Copyright (c) 2015-2017, Brook Tozer

All rights reserved.

Redistribution and use in source and binary forms, with or without modification,
are permitted provided that the following conditions are met:
\begin{itemize}
\item {} 
Redistributions of source code must retain the above copyright notice,
this list of conditions and the following disclaimer.

\item {} 
Redistributions in binary form must reproduce the above copyright notice,
this list of conditions and the following disclaimer in the documentation
and/or other materials provided with the distribution.

\item {} 
Neither the name of Brook Tozer nor the names of any contributors
may be used to endorse or promote products derived from this software
without specific prior written permission.

\end{itemize}

THIS SOFTWARE IS PROVIDED BY THE COPYRIGHT HOLDERS AND CONTRIBUTORS “AS IS” AND
ANY EXPRESS OR IMPLIED WARRANTIES, INCLUDING, BUT NOT LIMITED TO, THE IMPLIED
WARRANTIES OF MERCHANTABILITY AND FITNESS FOR A PARTICULAR PURPOSE ARE
DISCLAIMED. IN NO EVENT SHALL THE COPYRIGHT HOLDER OR CONTRIBUTORS BE LIABLE FOR
ANY DIRECT, INDIRECT, INCIDENTAL, SPECIAL, EXEMPLARY, OR CONSEQUENTIAL DAMAGES
(INCLUDING, BUT NOT LIMITED TO, PROCUREMENT OF SUBSTITUTE GOODS OR SERVICES;
LOSS OF USE, DATA, OR PROFITS; OR BUSINESS INTERRUPTION) HOWEVER CAUSED AND ON
ANY THEORY OF LIABILITY, WHETHER IN CONTRACT, STRICT LIABILITY, OR TORT
(INCLUDING NEGLIGENCE OR OTHERWISE) ARISING IN ANY WAY OUT OF THE USE OF THIS
SOFTWARE, EVEN IF ADVISED OF THE POSSIBILITY OF SUCH DAMAGE.


\subsection{Fatiando a Terra Licence}
\label{\detokenize{licence:fatiando-a-terra-licence}}
Copyright (c) 2010-2017, Leonardo Uieda

All rights reserved.

Redistribution and use in source and binary forms, with or without modification,
are permitted provided that the following conditions are met:
\begin{itemize}
\item {} 
Redistributions of source code must retain the above copyright notice,
this list of conditions and the following disclaimer.

\item {} 
Redistributions in binary form must reproduce the above copyright notice,
this list of conditions and the following disclaimer in the documentation
and/or other materials provided with the distribution.

\item {} 
Neither the name of Leonardo Uieda nor the names of any contributors
may be used to endorse or promote products derived from this software
without specific prior written permission.

\end{itemize}

THIS SOFTWARE IS PROVIDED BY THE COPYRIGHT HOLDERS AND CONTRIBUTORS “AS IS” AND
ANY EXPRESS OR IMPLIED WARRANTIES, INCLUDING, BUT NOT LIMITED TO, THE IMPLIED
WARRANTIES OF MERCHANTABILITY AND FITNESS FOR A PARTICULAR PURPOSE ARE
DISCLAIMED. IN NO EVENT SHALL THE COPYRIGHT HOLDER OR CONTRIBUTORS BE LIABLE FOR
ANY DIRECT, INDIRECT, INCIDENTAL, SPECIAL, EXEMPLARY, OR CONSEQUENTIAL DAMAGES
(INCLUDING, BUT NOT LIMITED TO, PROCUREMENT OF SUBSTITUTE GOODS OR SERVICES;
LOSS OF USE, DATA, OR PROFITS; OR BUSINESS INTERRUPTION) HOWEVER CAUSED AND ON
ANY THEORY OF LIABILITY, WHETHER IN CONTRACT, STRICT LIABILITY, OR TORT
(INCLUDING NEGLIGENCE OR OTHERWISE) ARISING IN ANY WAY OUT OF THE USE OF THIS
SOFTWARE, EVEN IF ADVISED OF THE POSSIBILITY OF SUCH DAMAGE.


\subsection{ObsPy Licence}
\label{\detokenize{licence:obspy-licence}}
GNU Lesser General Public License, Version 3 (\sphinxurl{https://www.gnu.org/copyleft/lesser.html})


\chapter{Release Notes}
\label{\detokenize{gmg_documentation:release-notes}}
Version: 1.0.0

Last updated: Aug 22, 2018


\chapter{Github}
\label{\detokenize{gmg_documentation:github}}
\fvset{hllines={, ,}}%
\begin{sphinxVerbatim}[commandchars=\\\{\}]
https://github.com/btozer/gmg
\end{sphinxVerbatim}


\chapter{Notices:}
\label{\detokenize{gmg_documentation:notices}}
Please send comments, feature requests and report all bugs to: \sphinxhref{mailto:btozer@ucsd.edu}{btozer@ucsd.edu}



\renewcommand{\indexname}{Index}
\printindex
\end{document}